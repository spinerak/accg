\documentclass[t]{beamer}
\usetheme[department=winuk,official=false]{tue2008}
\usepackage[english]{babel}
\usepackage{listings,amsmath,multimedia}

\graphicspath{{images/}}

\lstset{language=TeX,
        basicstyle=\color{black}\ttfamily,
        commentstyle=\color{gray}\it\ttfamily,
        keywordstyle=\color{tuered}\bf\ttfamily,
        showstringspaces=false,
        frame=single,
        backgroundcolor=\color{white},
        moretexcs={usetheme,frametitle,setbeamercovered,setbeameroption,usebackgroundtemplate,movie,logo,note,uncover,chapter,subsection,subsubsection,EUR,EURofc,includegraphics,lstset,color,it,bf,RequirePackage,pause,overlay,frontmatter,backmatter,mainmatter,maketitle,setlength,fancyhf,fancyhead,fancyfoot,lhead,chead,rhead,lfoot,cfoot,rfoot,texteuro,textcelsius,appendix,selectlanguage,part,tableofcontents}
}

\newenvironment{descrsf}[1]
  {\begin{list}{}{\renewcommand{\makelabel}[1]{\textsf{##1}\hfil}
                  \setlength{\itemsep}{0.5em}
                  \setlength{\parsep}{0pt}
                  \settowidth{\labelwidth}{\textsf{#1}}
                  \setlength{\labelsep}{10pt}
                  \setlength{\leftmargin}{\labelwidth}
                  \addtolength{\leftmargin}{\labelsep}
                  \providecommand{\descriptionlabel}[1]%
                      {\hspace{\labelsep}\textsf{#1}}
                 }
  }
  {\end{list}}

\title{\Huge CSGBuilder}
\author{Maarten Manders\\ Bart van Arnhem}


\begin{document}

\begin{frame}
\maketitle
\end{frame}

\setbeamercovered{transparent=30}
\begin{frame}
  \frametitle{Outline}
  \tableofcontents[pausesections]
\end{frame}
\setbeamercovered{invisible}

\section{Short recap of the project}
\begin{frame}
\frametitle{Short recap of the project}
\begin{itemize}
 \item Objects are implicit volumes
 \item Build complex objects from simple building block objects using CSG (\textbf{C}onstructive \textbf{S}olid \textbf{G}eometry)
 \item Constructed objects can be used as building blocks using save \& load
 \item Objects internally represented as CSG trees
 \item 
\end{itemize}
\end{frame}

\subsection{Constructive Solid Geometry}
\begin{frame}
\frametitle{Constructive Solid Geometry}
\begin{itemize}
 \item Boolean operations like union, difference and intersection can be
 realized using simple boolean operators on the implicit functions.
\end{itemize}
\begin{figure}
\includegraphics<1->[width=0.4\textwidth]{accgimg/Boolean_difference}
\caption{Source http://en.wikipedia.org/wiki/Constructive\_solid\_geometry}
\end{figure}
\end{frame}

\subsection{CSG Trees}
\begin{frame}
\frametitle{CSG Trees}
\begin{itemize}
 \item \textbf{C}onstructive \textbf{S}olid \textbf{G}eometry tree
\end{itemize}
\begin{figure}
\includegraphics<1->[width=0.4\textwidth]{accgimg/Csg_tree}
\caption{Source http://en.wikipedia.org/wiki/Constructive\_solid\_geometry}
\end{figure}
\end{frame}

\subsection{Marching Cubes}
\begin{frame}
\frametitle{CSG Trees}
\begin{itemize}
 \item Adaptive variant using octrees
 \item Subdivide cube into 8 cubes where needed (for example wherever
  the object has sharp features)
 \item Detect sharp features using angle between surface normals
\end{itemize}
\end{frame}

\section{Project status}
\begin{frame}
\frametitle{Project status}
- done
\end{frame}

\section{Demo}
\begin{frame}
\frametitle{Demo}
\end{frame}

\section{Questions?}
\begin{frame}
\frametitle{Questions?}
\end{frame}

\end{document} 