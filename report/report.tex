\documentclass[a4paper,10pt,twoside]{report}
\usepackage{a4wide,url,longtable,lscape,verbatim,amssymb,graphicx,multirow,fancyhdr,lastpage,algorithm2e}

% --- Informal Requirements Commands ---
\newcounter{req_count}
\setcounter{req_count}{1}


\newcommand{\course}{Additional Component Computer Graphics (2IV05)}
\newcommand{\student}{\footnotesize Page \thepage~of \pageref{LastPage}}
\newcommand{\subject}{Final report}

\fancypagestyle{plain}{%
\fancyhf{} %
\fancyfoot[C]{\student} %
\fancyhead[LE, RO]{\subject} %
\fancyhead[LO, RE]{\course} %
}

\pagestyle{fancy}{%
\fancyhf{} %
\fancyfoot[C]{\student} %
\fancyhead[LE, RO]{\subject} %
\fancyhead[LO, RE]{\course} %

% Used commands
\newcommand{\Bound}{\mbox{Bound}}

% Nice algorithm styles
\linesnumbered
\dontprintsemicolon
\SetLine

% No paragraph indent, it's ugly
\setlength\parindent{0em}

% Make sure no text goes beyond the page
\setlength{\textheight}{60em}

%opening
\title{2IV05 - Additional Component Computer Graphics \\ Final Report}
\author{Bart van Arnhem (0557184) \\ Maarten Manders (0573419)}

\begin{document}

\maketitle

\chapter{Introduction}
This report describes the work that we have done for the Additional Component Computer Graphics course (2IV05). We have written a program called \textit{CSGBuilder} which allows a user to create complex three-dimensional objects using only a small number of simple primitives, which are basically simple objects.\\

We have used a technique, called Constructive Solid Geometry, or CSG in short, to create our models. This technique uses mathematical functions to represent primitives and boolean operators to construct extended objects using these primitives.\\

For the rendering of our objects we used the adaptive marching cubes algorithm. This algorithm divides the space around the CSG objects in a number of cubes and constructs a mesh that approaches the form of the object. We have used the adaptive form of this algorithm, because this increases the precision of the rendered objects.\\

The rest of this document will describe the techniques that were mentioned above in more detail. Also the way we used the techniques will be explained. Besides this, the architecture and user interface of our program are explained.\\

We have used the Java\textcopyright\ programming language in combination with Swing and the Java OpenGL\texttrademark\ API (JOGL) to write our program. To increase the responsiveness of our program we used the JOGL AWT GLCanvas instead of the Swing GLJPanel, because the drawing on the AWT component is faster.

\tableofcontents

\chapter{Requirements}
This chapter is divided in two sections, in which different aspects of our system are specified. The ``Functionality'' section describes the functionality of CSGBuilder. The ``Interaction and presentation'' section describes the ways how the users of our software can interact with CSGBuilder.

\section{Functionality}
This section describes the functionality of the program. There are two kinds of functionality that will be added. ``Core functionality'' is that part of the software that must be implemented in order to make it work properly. ''Extra functionality'' is that part of the software that makes it more appealing. If time permits, all the functionality will be implemented, otherwise parts of the extra functionality will be dropped.\\

The core functionality consists of: \\
\vspace{-10pt}
\begin{enumerate}
	\item Object editing window
	\item Object selector window
	\item Object loading
	\item Object saving
	\item Shape Union
	\item Shape intersection
	\item Shape difference
	\item Zooming
	\item Rotation
\end{enumerate}
\bigskip
The extra functionality consists of: \\
\vspace{-10pt}
\begin{enumerate}
	\item Texturing
	\item Adding temporary shapes
	\item Preview
	\item Lighting
\end{enumerate}

\subsection{Possible extra functionality (if there is time)}
Possibility to specify a ``weight'', meaning only a certain amount of material is added or removed from the geometric object.

\section{Interaction and presentation}

\begin{figure}[h]
\centering
\includegraphics[width = 0.8\textwidth]{../resources/gui_mockup/1}
\caption{\emph{Mockup of the graphical user interface}} \label{gui_mockup}
\end{figure}

Figure \ref{gui_mockup} displays a mockup of the graphical user interface. The right and left frame contain respectively the ``building block'' and the ``result'' models. Using the mouse both models can be zoomed an rotated in an intuitive way (like in 3D Studio Max).\\

Clicking with the left mouse button in the left frame will result in the model loaded in the right frame being added to the result model at the clicked position. Using the right mouse button the model can be removed from the result model. \\

Both models can be loaded from a file using the ``Load model'' buttons. The result model can also be saved to a file. \\

To give the user an idea of where the material is going to be added or removed a semi transparent version of the ``building block'' model is rendered at the mouse cursor.\\

\chapter{Representing Objects}

\section{CSG Tree}
    To represent the objects that are modeled with our program, the Constructive Solid Geometry (CSG) technique is used. This technique uses a number of simple objects, called primitives, that are combined in to more complex objects by using boolean operators. To represent the CSG objects the CSG trees that are described in \cite{Wiegand96} are used. Figure~\ref{figure:csg_tree} shows an example of a simple CSG tree.\\

    \begin{figure}[h]
        \begin{center}
            \includegraphics[scale=.2]{./images/csgtree.png}
        \end{center}
        \caption{An example of a CSG tree (source: wikipedia\cite{img:wiki_csg_tree}).}
        \label{figure:csg_tree}
    \end{figure}

    In a CSG tree the boolean operators are the nodes in a binary tree and the primitives are the leaves of the tree. The primitives are represented by mathematical functions in $\mathbb{R}^3$. The operators combine the primitives in a object by using boolean functions on the results of the primitive functions. How this is done will be explained in the next sections.\\

\subsection{Primitives}
    A primitive in a CSG tree is a function $A:\mathbb{R}^3 \rightarrow \mathbb{R}$. The form of an object is constructed by drawing the points $(x,y,z)$ for which $A(x,y,z)=0$. Our program has a small number of primitives, these are described in Table~\ref{table:primitives}. In this table $p_x$, $p_y$ and $p_z$ denote the coordinates of the position of the center of the primitive and $r_x$, $r_y$ and $r_z$ denote the size of the primitive in respectively the $x$, $y$ and $z$ direction.\\

    \begin{table}[h]
        \begin{longtable}{|p{.1\linewidth}|p{.45\linewidth}|p{.45\linewidth}|}
            \hline
            \textbf{Primitive} & \textbf{Description} & \textbf{Formula}\\
            \hline
            \hline
             Ellipsoid &
             Spheres or cigar formed shapes &
             $f(x,y,z)=(\frac{p_x + x}{r_x})^2+(\frac{p_x + x}{r_x})^2+(\frac{p_x + x}{r_x})^2 - 1$\\
            \hline
             Cuboid &
             Rectangles or bars with rounded corners &
             $f(x,y,z)=(\frac{p_x + x}{r_x})^4+(\frac{p_x + x}{r_x})^4+(\frac{p_x + x}{r_x})^4 - 1$\\
            \hline
        \end{longtable}
        \caption{The primitives of CSGBuilder}
        \label{table:primitives}
    \end{table}

    These primitives can be rotated with arbitrary angels around all axes by multiplying a rotation matrix $R$, which is shown in Figure~\ref{figure:rotation_matrix}, with the vector $\overrightarrow{v}=(x,y,z)^T$, so the rotated point $\overrightarrow{w}=R\overrightarrow{v}$. Applying a primitive function $A$ on $\overrightarrow{w}^T$ causes the primitive to be rotated with angles $\alpha_x$, $\alpha_y$ and $\alpha_z$ around respectively the $x$, $y$ and $z$ axes.\\

    \begin{figure}[h]
        \[
            \left(
                \begin{array}{lll}
                    \cos(\alpha_y)\cos(\alpha_z) &
                    \cos(\alpha_y)\sin(\alpha_z) &
                    -\sin(\alpha_y) \\

                    \sin(\alpha_x)\sin(\alpha_y)\cos(\alpha_z) - \cos(\alpha_x)\sin(\alpha_z) &
                    \sin(\alpha_x)\sin(\alpha_y)\sin(\alpha_z) + \cos(\alpha_x)\cos(\alpha_z) &
                    \sin(\alpha_x)\cos(\alpha_y) \\

                    \cos(\alpha_x)\sin(\alpha_y)\cos(\alpha_z) + \sin(\alpha_x)\sin(\alpha_z) &
                    \cos(\alpha_x)\sin(\alpha_y)\sin(\alpha_z) - \sin(\alpha_x)\cos(\alpha_z) &
                    \cos(\alpha_x)\cos(\alpha_y) \\
                \end{array}
            \right)
        \]
        \caption{The rotation matrix $R$}
        \label{figure:rotation_matrix}
    \end{figure}

\subsection{Operators}
    There are three operations that can be used to construct a CSG tree. These are the union($\cup$), intersection($\cap$) and difference($-$) set operations. These set operations have to be mapped to the mathematical functions of which CSG trees are constructed, this can be done using the $\min$ and $\max$ functions, as can be seen in Figure~\ref{figure:operations}. In the figure $A, B:\mathbb{R}^3 \rightarrow \mathbb{R}$ are functions (possibly CSG trees) of which a CSG tree is constructed.\\

    \begin{figure}[h]
            \begin{longtable}{lll}
                \textbf{Union: } &
                $A \cup B$ &
                $\forall[x,y,z \in \mathbb{R} :: \min(A(x, y, z), B(x,y,z))]$ \\

                \textbf{Intersection: } &
                $A \cap B$ &
                $\forall[x,y,z \in \mathbb{R} :: \max(A(x, y, z), B(x,y,z))]$ \\

                \textbf{Difference: } &
                $A - B$ &
                $\forall[x,y,z \in \mathbb{R} :: \max(A(x, y, z), -B(x, y, z))]$ \\
            \end{longtable}
        \caption{The operations to construct CSG trees.}
        \label{figure:operations}
    \end{figure}

    By combining the primitives with the operators, it is possible to create complex objects. When these objects consist of many primitives and operators the rendering of the object tends to be slow. It is possible increase the rendering speed by creating a normalize CSG tree. A way in which CSG trees can be normalized is described in the next section.\\

\subsection{Normalizing CSG trees}
    As described in \cite{Wiegand96}, it is possible to increase the rendering speed, by normalizing CSG trees to a \textit{sum of products} form. This means that that all intersection and difference operators have a left subtree which contains no union operators and a right subtree that consists of a primitive. To create a normalized CSG tree, the nine operations that are described in \cite{Wiegand96} are used. These operations are recalled in Figure~\ref{table:normalize}.\\

    \begin{table}[h]
        \begin{longtable}{llll}
            1. & $X - (Y \cup Z)$    & $\rightarrow$ & $(X - Y) - Z$\\
            2. & $X \cap (Y \cup Z)$ & $\rightarrow$ & $(X \cap Y) \cup (X \cap Z)$\\
            3. & $X - (Y \cap Z)$    & $\rightarrow$ & $(X - Y) \cup (X - Z)$\\
            4. & $X \cap (Y \cap Z)$ & $\rightarrow$ & $(X \cap Y) \cap Z$\\
            5. & $X - (Y - Z)$       & $\rightarrow$ & $(X - Y) \cup (X \cap Z)$\\
            6. & $X \cap (Y - Z)$    & $\rightarrow$ & $(X \cap Y) - Z$\\
            7. & $(X - Y) \cap Z$    & $\rightarrow$ & $(X \cap Z) - Y$\\
            8. & $(X \cup Y) - Z$    & $\rightarrow$ & $(X - Z) \cup (Y - Z)$\\
            9. & $(X \cup Y) \cap Z$ & $\rightarrow$ & $(X \cap Z) \cup (Y \cap Z)$\\
        \end{longtable}
        \caption{Operations to normalize a CSG tree.}
        \label{table:normalize}
    \end{table}

    When the normalization is completed, the CSG tree might contain nodes which could be rendered even faster. The normalized CSG tree could be pruned according to the operations that are stated in Table~\ref{table:prune}. The pruning of the tree leads to less calculations, so the time that is needed to render the object is decreased.\\

    \begin{table}[h]
        \begin{longtable}{lll}
            1. & $A \cap B \rightarrow \emptyset$ & if $\Bound(A)$ does not intersect $\Bound(B)$\\
            2. & $A - B \rightarrow A$            & if $\Bound(A)$ does not intersect $\Bound(B)$\\
        \end{longtable}
        \caption{Operations to prune a normalized CSG tree.}
        \label{table:prune}
    \end{table}

    In order to decide whether a node in the tree should be pruned, the cases in which the pruning takes place should be recognized. This can be done by constructing and calculating the bounding boxes of nodes in the CSG tree. The bounding box $\Bound$ of a primitive is simply the smallest cube in which the primitive can be placed. To determine the bounding box of operations in the CSG tree the equations in Table~\ref{table:bounding_box} can be used.\\

    \begin{table}[h]
        \begin{longtable}{llll}
            1. & $\Bound(A \cup B)$ & $=$ $\Bound(\Bound(A) \cup \Bound(B))$\\
            2. & $\Bound(A \cap B)$ & $=$ $\Bound(\Bound(A) \cap \Bound(B))$\\
            3. & $\Bound(A - B)$    & $=$ $\Bound(A)$\\
        \end{longtable}
        \caption{The computation of a bounding box of a CSG tree.}
        \label{table:bounding_box}
    \end{table}

    Applying the normalization and pruning operations on the nodes of a tree leads to the normalization algorithm that is described in  Figure~\ref{figure:algorithm}. This algorithm is based on the normalization algorithm in \cite{Wiegand96}.\\
    
    \begin{figure}[ht]
            \textbf{Algorithm} $\textsc{Normalize}(T: CSGTree)$\\
            \noindent
            \begin{algorithm}[H]
                \If{$T$ is a primitive}{
                    \Return\;
                }
                \BlankLine
                \Repeat{($T\mbox{.op is a union}) \lor ((T\mbox{.right is a primitive}) \land (T\mbox{.left is not a union}))$}{
                    \Repeat{$\neg changed$}{
                        $changed \gets \textsc{False}$\;
                        \If{$T$ matches a rule from Figure~\ref{table:normalize}} {
                            Apply the first matching rule\;
                            $changed \gets \textsc{True}$\;
                        }
                    }
                    \textsc{Normalize}$(T\mbox{.left})$\;
                    \textsc{Prune}$(T)$\;
                }
                \BlankLine
                \textsc{Normalize}$(T\mbox{.right})$\;
            \end{algorithm}
        \caption{The algorithm that generates a normalized CSG tree}
        \label{figure:algorithm}
    \end{figure}
    
\chapter{Rendering Objects}
\section{Adaptive Marching Cubes}

\chapter{Architecture}

\chapter{Evaluation}

\chapter{Conclusion}

\bibliographystyle{amsplain}
\begin{thebibliography}{99}
\bibitem{Wiegand96} 
    T.~F.~Wiegand.\\
    Interactive Rendering of CSG Models.\\
    The Eurographics Association, 1996.
    
\bibitem{img:wiki_csg_tree}
    Unknown author.\\
    Csg tree.png.\\
    http://en.wikipedia.org/wiki/File:Csg\_tree.png.\\
\end{thebibliography}
\end{document}